
\operative{Calls}{upon the parties concerned to implement immediately its resolution 338 (1973) of 22 October 1973}
\operative{Stresses}{the obligation on both parties to scrupulously and fully respect the terms of the 1974 Disengagement of Forces Agreement, calls on the parties to exercise maximum restraint and prevent any breaches of the ceasefire and the area of separation, encourages the parties to take full advantage of UNDOF’s liaison function regularly to address issues of mutual concern, as appropriate, and to maintain their liaison with UNDOF to prevent any escalation of the situation across the ceasefire line, as well as to support the enhancement of the UNDOF liaison function, and underscores that there should be no military activity of any kind in the area of separation, including military operations by the Syrian Arab Armed Forces}
\operative{Underlines}{that UNDOF remains an impartial entity and stresses the importance to halt all activities that endanger United Nations peacekeepers on the ground and to accord the United Nations personnel on the ground the freedom to carry out their mandate safely and securely}
\operative{Expresses}{full support for Major General Nirmal Kumar Thapa as Head of Mission and Force Commander}
\operative{Calls}{on all groups other than UNDOF to abandon all UNDOF positions, and return the peacekeepers’ vehicles, weapons, and other equipment}
\operative{Calls}{on all parties to cooperate fully with the operations of UNDOF, to respect its privileges and immunities and to ensure its freedom of movement, as well as the security of and unhindered and immediate access for the United Nations personnel carrying out their mandate, including the unimpeded delivery of UNDOF equipment and the temporary use of alternative ports of entry and departure, as required, to ensure safe and secure troop rotation and resupply activities, in conformity with existing agreements, and urges prompt reporting by the Secretary- General to the Security Council and troop-contributing countries of any actions that impede UNDOF’s ability to fulfil its mandate}
\operative{Calls}{on the parties to provide all the necessary support to allow for the full utilization of the Quneitra crossing by UNDOF in line with established procedures and to lift COVID-19 related restrictions as soon as sanitary conditions permit, to allow UNDOF to increase its operations on the Bravo side to facilitate effective and efficient mandate implementation}
\operative{Requests}{UNDOF, within existing capacities and resources, member states, and relevant parties to take all appropriate steps to protect the safety, security and health of all UNDOF personnel, in line with resolution 2518 (2020), taking into account the impact of the COVID-19 pandemic}
\operative{Welcomes}{UNDOF’s ongoing efforts to consolidate its presence and to intensify its operations in the area of separation, including the mission’s intent to resume inspections in all areas of limitation on the Bravo side, conditions permitting per the Mission’s assessment, as well as the cooperation of the parties to facilitate this return, together with continued efforts to plan for UNDOF’s expeditious return to the area of separation, including the provision of adequate force protection, based on a continuous assessment of security in the area}

\operative{Underscores}{the importance of progress in the deployment of appropriate technology, including counter-improvised explosive device (IED) capabilities and a sense and warn system, as well as in addressing civilian staffing needs, to ensure the safety and security of UNDOF personnel and equipment, following appropriate consultations with the parties, and notes in this regard that the Secretary-General’s proposal for such technologies has been delivered to the parties for approval}

\operative{Encourages}{the parties to the Disengagement Agreement to engage constructively to facilitate necessary arrangements with UNDOF for the force’s return to the area of separation, taking into account existing agreements}

\operative{Encourages}{the Department of Peace Operations, UNDOF, and the UN Truce Supervision Organization to continue relevant discussions on recommendations from the 2018 independent review to improve mission performance and implementation of UNDOF’s mandate}

\operative{Welcomes}{the initiatives undertaken by the Secretary-General to standardize a culture of performance in UN peacekeeping, recalls its request in resolution 2378 (2017) and resolution 2436 (2018) that the Secretary-General ensure that performance data related to the effectiveness of peacekeeping operations is used to improve mission operations, including decisions such as those regarding deployment, remediation, repatriation and incentives, and reaffirms its support for the development of a comprehensive and integrated performance policy framework that identifies clear standards of performance for evaluating all United Nations civilian and uniformed personnel working in and supporting peacekeeping operations that facilitates effective and full implementation of mandates, and includes comprehensive and objective methodologies based on clear and well-defined benchmarks to ensure accountability for underperformance and incentives and recognition for outstanding performance, and calls on the United Nations to apply this framework to UNDOF as described in resolution 2436 (2018), notes the efforts of the Secretary-General to develop a comprehensive performance assessment system and requests the Secretary-General and troop- and police-contributing countries to seek to increase the number of women in UNDOF, as well as to ensure the full, equal, and meaningful participation of uniformed and civilian women at all levels, and in all positions, including senior leadership positions, and to implement other relevant provisions of resolution 2538 (2020)}
\operative{Requests}{the Secretary-General to continue to take all necessary measures to ensure full compliance of all personnel in UNDOF, civilian and uniformed, including mission leadership and mission support personnel with the United Nations zero-tolerance policy on sexual exploitation and abuse and to keep the Council fully informed through his reports to the Council about the Mission’s progress in this regard, including by reporting on the start, agreed deadlines, and outcomes of 2272 reviews, stresses the need to prevent such exploitation and abuse and to improve how these allegations are addressed in line with resolution 2272 (2016), and urges troop- and police-contributing countries to continue taking appropriate preventive action, including vetting of all personnel, pre-deployment and in-mission awareness training, and to take appropriate steps to ensure full accountability in cases of such conduct involving their personnel through timely investigation of allegations by troop- and police-contributing countries, and UNDOF as appropriate, holding perpetrators to account and repatriating units when there is credible evidence of widespread or systemic sexual exploitation and abuse by those units}
\operative{Decides}{to renew the mandate of UNDOF for a period of six months, that is, until 30 June 2023, and requests the Secretary-General to ensure that UNDOF has the required capacity and resources to fulfil the mandate in a safe and secure way}
\operative{Requests}{the Secretary-General to report every 90 days on developments in the situation and the measures taken to implement resolution 338 (1973)}
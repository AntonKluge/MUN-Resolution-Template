\pream{Noting}{with concern that the situation in the Middle East is tense and is likely to remain so, unless and until a comprehensive settlement covering all aspects of the Middle East problem can be reached}

\pream{Having}{considered the reports of the Secretary-General on the United Nations Disengagement Observer Force (UNDOF) of 22 September 2022 (S/2022/711) and 29 November 2022 (S/2022/887), and also reaffirming its resolution 1308 (2000) of 17 July 2000}

\pream{Stressing}{that both parties must abide by the terms of the 1974 Disengagement of Forces Agreement between Israel and the Syrian Arab Republic and scrupulously observe the ceasefire}

\pream{Expressing}{concern that the ongoing military activities conducted by any actor in the area of separation continue to have the potential to escalate tensions between Israel and the Syrian Arab Republic, jeopardize the ceasefire between the two countries, and pose a risk to the local civilian population and United Nations personnel on the ground}

\pream{Expressing}{its appreciation in this regard for UNDOF’s liaison efforts to prevent any escalation of the situation across the ceasefire line}

\pream{Expressing}{alarm that violence in the Syrian Arab Republic risks a serious conflagration of the conflict in the region}

\pream{Expressing}{concern at all violations of the Disengagement of Forces Agreement}

\pream{Noting}{the Secretary-General’s latest report on the situation in the area of operations of UNDOF, including findings about weapons fire across the ceasefire line as well as ongoing military activity on the Bravo side of the area of separation, and in this regard, underscoring that there should be no military forces, military equipment, or personnel in the area of separation other than those of UNDOF}

\pream{Calling}{upon all parties to the Syrian domestic conflict to cease military actions throughout the country, including in the UNDOF area of operations, and to respect international humanitarian law}

\pream{Noting}{the significant threat to the UN personnel in the UNDOF area of operation from unexploded ordnance, explosive remnants of war and mines, and emphasizing in this regard the need for demining and clearance operations in strict compliance with the 1974 Disengagement of Forces Agreement}

\pream{Reaffirming}{its readiness to consider listing individuals, groups, undertakings, and entities providing support to ISIL(Da’esh) or to the Al-Nusra Front (also known as Jabhat Fateh al-Sham or Hay’at Tahrir al-Sham), including those who are financing, arming, planning, or recruiting for ISIL (Da’esh) or the Al-Nusra Front and all other individuals, groups, undertakings, and entities associated with ISIL (Da’esh) and Al-Qaida as listed on the ISIL (Da’esh) and Al-Qaida Sanctions List, including those participating in or otherwise supporting attacks against UNDOF peacekeepers}

\pream{Recognizing}{the necessity of efforts to flexibly adjust UNDOF’s posture to minimize the security risk to UNDOF personnel as UNDOF continues to implement its mandate, while emphasizing that the ultimate goal is for the peacekeepers to return to UNDOF’s area of operations as soon as practicable}

\pream{Emphasizing}{the importance of Security Council and troop-contributing countries having access to reports and information related to UNDOF’s redeployment configuration, and reinforcing that such information assists the Security Council with evaluating, mandating, and reviewing UNDOF and with effective consultation with troop-contributing countries}

\pream{Underscoring}{the need for UNDOF to have at its disposal all necessary means and resources to carry out its mandate safely and securely, including technology and equipment to enhance its observation of the area of separation and the ceasefire line, and to improve force protection, as appropriate, and recalling that the theft of United Nations weapons and ammunition, vehicles and other assets, and the looting and destruction of United Nations facilities, are unacceptable}

\pream{Expressing}{its profound appreciation to UNDOF’s military and civilian personnel, including those from Observer Group Golan, for their service in an ongoing, challenging operating environment, underscoring the important contribution UNDOF’s continued presence makes to peace and security in the Middle East, welcoming steps taken to enhance the safety and security of UNDOF, including Observer Group Golan, personnel, and stressing the need for continued vigilance to ensure the safety and security of UNDOF and Observer Group Golan personnel}

\pream{Strongly condemning}{incidents threatening the safety and security of United Nations personnel}

\pream{Expressing}{its appreciation to UNDOF, including Observer Group Golan, for progress towards expanding its presence in its area of operations through patrols and rehabilitation of positions on the Bravo side}

\pream{Taking note}{of the Secretary-General’s plan for UNDOF to return to the Bravo side based on a continuous assessment of security in the area of separation and its surroundings, and continued discussion and coordination with the parties}

\pream{Recalls}{that UNDOF’s deployment and the 1974 Disengagement of Forces Agreement are steps toward a just and durable peace on the basis of Security Council Resolution 338 (1973)}

\pream{Recalling}{resolution 2378 (2017) and its request of the Secretary-General to ensure that data related to the effectiveness of peacekeeping operations, including peacekeeping performance data, is used to improve analytics and the evaluation of mission operations, based on clear and well identified benchmarks, and further recalling resolution 2436 (2018) and its request of the Secretary-General to ensure that decisions to recognize and incentivize outstanding performance and decisions regarding deployment, remediation, training, withholding of financial reimbursement, and repatriation of uniformed or dismissal of civilian personnel, are predicated on objective performance data}

\pream{Recalling}{resolution 2242 (2015) and its aspiration to increase the number of women in military and police contingents of United Nations peacekeeping operations}
